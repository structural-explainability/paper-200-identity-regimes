% !TeX root = se200_identity_regimes.tex
\section{Sufficiency: A Six-Regime Construction}
\label{sec:sufficiency}

This section shows that the lower bound established in
Section~\ref{sec:necessity} is tight.
We exhibit a construction with a minimal set of six distinct
identity-and-persistence regimes that satisfies all structural requirements
stated in Section~\ref{sec:requirements}, while respecting the neutrality and
stability constraints fixed in Section~\ref{sec:assumptions}.
No additional regimes or hidden interpretive mechanisms are required.

\subsection{The Six Identity-and-Persistence Regimes}

We realize six pairwise-disjoint identity-and-persistence regimes.
For concreteness, each regime is defined solely by its 
identity conditions, persistence behavior, and admissible relations:

\begin{itemize}
  \item \textbf{K1 (Actors):} enduring entities capable of bearing obligations,
        rights, or responsibilities.
  \item \textbf{K2 (Loci/Assets):} enduring physical or operational entities that
        may be acted upon but do not bear obligations.
  \item \textbf{K3 (Instruments):} enduring authority-bearing structures that
        ground obligations or permissions independently of any particular
        occurrence.
  \item \textbf{K4 (Occurrences):} time-indexed actions or events whose identity
        is inseparable from temporal realization and provenance.
  \item \textbf{K5 (Scopes):} enduring applicability contexts that delimit where,
        to whom, or under what conditions instruments apply.
  \item \textbf{K6 (Indicators):} descriptive records or measurements that assert
        observed or derived facts without asserting causal or normative claims.
\end{itemize}

Each regime fixes a distinct identity criterion that is invariant under
incompatible extension.
No regime changes over time, and no regime’s identity depends on
interpretation supplied by analytic or application layers.

\subsection{Admissible Relations}

A minimal set of typed, directional relations suffices to connect the six
regimes and to satisfy all stated requirements.
Representative relations include:

\begin{itemize}
  \item authority grounding and delegation:
        $\textit{enacts}, \textit{issues} : \text{K1} \rightarrow \text{K3}$
  \item participation:
        $\textit{party-to} : \text{K1} \rightarrow \text{K3}$
  \item execution under authority:
        $\textit{occurs-under} : \text{K4} \rightarrow \text{K3}$
  \item involvement and action:
        $\textit{involves} : \text{K4} \rightarrow \text{K1}$,
        $\textit{acts-on} : \text{K4} \rightarrow \text{K2}$
  \item applicability:
        $\textit{applies-in} : \text{K3} \rightarrow \text{K5}$
  \item description:
        $\textit{measures} : \text{K6} \rightarrow
        (\text{K1} \mid \text{K2} \mid \text{K5})$
\end{itemize}

This relation set is not exhaustive, but illustrates that no additional
identity regimes are required to satisfy the structural constraints.
Temporal and provenance information attach only to K4 and K6 via attribute
schemas; no separate entities for time, provenance, or explanation are
introduced.

\subsection{Satisfaction of Structural Requirements}

We briefly verify that the construction satisfies the requirements of
Section~\ref{sec:requirements}:

\begin{itemize}
  \item \textbf{R1 (Stable identity):} each regime fixes a distinct, invariant
        identity criterion under extension.
  \item \textbf{R2 (Obligation-bearing):} obligations attach only to K1 via K3,
        preserving role stability.
  \item \textbf{R3 (Normative reference):} K3 grounds authority independently of
        execution, without causal or evaluative claims.
  \item \textbf{R4 (Time-indexed occurrence):} K4 captures temporal action without
        persistence ambiguity.
  \item \textbf{R5 (Applicability):} K5 provides first-class scope referents that
        support nesting and overlap without reclassification.
  \item \textbf{R6 (Descriptive records):} K6 represents indicators as stable
        referents distinct from occurrences, enabling longitudinal comparison
        without causal commitment.
\end{itemize}

All requirements are satisfied without introducing hidden regimes,
role-based typing, or interpretive predicates at the substrate level.

\subsection{Sufficiency Proposition}

\begin{proposition}[Sufficiency of six regimes]
\label{prop:sufficiency-six}
The six-regime construction defined above satisfies the neutrality, stability,
and accountability requirements fixed in
Sections~\ref{sec:assumptions} and~\ref{sec:requirements}.
\end{proposition}

\begin{proof}
Each requirement is satisfied by a distinct identity-and-persistence regime
with explicitly constrained admissible relations.
No requirement demands an additional regime, and no regime is redundant under
the stated optimization criterion.
Because the construction preserves stable reference under incompatible
extension and introduces no hidden distinctions, it is sufficient.
\end{proof}

\subsection{Tightness of the Bound}

Together with the necessity result of Section~\ref{sec:necessity}, the
sufficiency proposition establishes that six identity-and-persistence regimes
are both necessary and sufficient for a neutral accountability substrate under
the stated assumptions.
The bound is therefore tight with respect to the optimization objective of
interoperability under persistent disagreement.
