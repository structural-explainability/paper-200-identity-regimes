% !TeX root = se200_identity_regimes.tex
\section{Introduction: Structure Under Disagreement}
\label{sec:intro}

Modern data systems increasingly operate in environments characterized by
persistent legal, political, and analytic disagreement.
In such settings, no single interpretive authority can be assumed, and
semantic convergence cannot be enforced without excluding legitimate
participants or perspectives.
Nevertheless, shared use of data remains necessary:
entities must be referable, actions traceable, and
records comparable across frameworks that may diverge
in assumptions, methods, or evaluative standards.

This paper addresses a structural question that arises in this context:
\emph{what ontological structure is required if a representation is to remain
usable as a shared substrate across incompatible interpretations?}
The focus is not on domain modeling, explanatory adequacy, or normative correctness.
Rather, it concerns the structural conditions required for stable reference
and accountability under persistent disagreement.

We adopt a neutrality framework developed in prior work~\cite{case2025neutrality},
treating ontological neutrality as a design constraint rather than an
aspirational ideal.
Neutrality is understood here as interpretive non-commitment combined with
stability under incompatible extension.
Under this view, a substrate must permit independent extensions to introduce
causal, normative, or analytic structure without requiring revision of the
underlying representation.
Accordingly, explanatory and evaluative structure is externalized, and the
substrate is constrained to remain pre-causal and pre-normative at the
substrate level.

Within this setting, ontological minimality cannot be understood solely as a
reduction in the number of primitives.
Many apparent simplifications achieve parsimony by reintroducing essential
distinctions indirectly, through roles, flags, contextual typing, or other
internal discriminators whose interpretation is supplied by extension layers
(e.g., a single undifferentiated entity type distinguished only by role
predicates such as \emph{isAgent}, \emph{isOccurrence}, or \emph{isAuthority}).
Such encodings rely on shared interpretive assumptions, which may not be
available in settings characterized by persistent disagreement.
For this reason, the present analysis excludes reductions that preserve
stability only by deferring essential distinctions to interpretive mechanisms
not represented at the substrate level.

The specific focus of this paper is accountability.
Accountability analysis requires stable reference to participants, authority
structures, regulated occurrences, applicability contexts, and descriptive
records, without embedding causal claims or normative judgments in the
substrate itself.
These categories are not introduced as an ontological partition,
but as representational capacities whose simultaneous support
places structural constraints on any neutral substrate.
The question addressed here is not how these elements should be interpreted,
but whether a neutral substrate can support them at all, and if so, with what
irreducible structural commitments.

The central contribution of this paper is a conditional bound on ontological
structure given the design goals outlined above.
Under explicit neutrality and stability constraints, we show that any ontology
capable of supporting accountability under persistent disagreement must
realize at least six distinct identity-and-persistence regimes.
We further show that a construction with a minimal set of six such regimes is
sufficient to satisfy the stated requirements without introducing hidden or
implicit structure.
The result is not a proposal for a universal ontology, but a constraint on
what is possible under a specific optimization objective.

Alternative optimization objectives prioritize different tradeoffs, such as
interpretive convergence, domain-specific efficiency, or context-dependent
classification at the substrate level.
The results presented here apply only under the stated constraints, within
which a minimal, domain-independent bound on ontological structure is
established for stable reference and accountability.

The remainder of the paper proceeds as follows.
Section~\ref{sec:assumptions} states the assumptions, neutrality constraints,
and optimization criterion.
Section~\ref{sec:requirements} formalizes the structural requirements imposed
by accountability analysis.
Section~\ref{sec:necessity} derives a lower bound on the number of required
identity-and-persistence regimes.
Section~\ref{sec:sufficiency} presents a construction achieving this bound.
Section~\ref{sec:interpretation} discusses the scope and implications of the
result.
Section~\ref{sec:related} situates the work relative to existing ontology
frameworks.
Section~\ref{sec:conclusion} concludes.
