% !TeX root = se200_identity_regimes.tex
\section{Assumptions, Neutrality, and Optimization Criterion}
\label{sec:assumptions}

This section fixes the problem setting for the remainder of the paper.
All results that follow are conditional on the assumptions stated here.
The purpose is not to argue that these assumptions are universally correct,
but to make explicit the constraints under which a neutral substrate for
accountability is possible.

\subsection{Persistent Disagreement and Uncoordinated Extension}

We assume environments in which disagreement is not an anomaly to be
resolved, but a persistent condition.
Legal interpretation, policy evaluation, and analytic methodology may remain
incompatible across communities and over time.
Accordingly, no single interpretive authority or shared semantic framework
can be assumed at the substrate level.

We further assume that extensions to the substrate may occur without
coordination.
Independent actors may add explanatory, normative, or analytic structure,
and such extensions must not require revision of the substrate itself.
This assumption does not rely on 
global semantic agreement, centralized governance, or 
negotiated meaning as preconditions for interoperability.

\subsection{Neutrality Constraint}

This work assumes substrate neutrality and stability 
as a design constraint~\cite{case2025neutrality} 
rather than as a philosophical claim.
A neutral substrate does not assert why
events occur, whether outcomes are desirable, or 
how actions ought to be evaluated.
Instead, it provides stable referents and structural relations upon which
external interpretive layers may operate.

Neutrality further requires stability under incompatible extension.
If two extensions disagree in interpretation, evaluation, or explanation,
their disagreement must not force reclassification of entities, alteration
of identity conditions, or revision of the substrate.
Any representation whose admissibility conditions depend on interpretive
agreement does not satisfy this constraint.

\subsection{Stable Reference Requirement}

For a substrate to support accountability, entities must remain referable
over time and across extensions.
Identity conditions and admissible relationships must be invariant under
changes in interpretation.
An entity may participate in new relationships or acquire additional
annotations, but it must not change its ontological status in order to remain
usable as a shared referent.

This requirement excludes representations in which stability-critical
distinctions are encoded only as roles, flags, or contextual predicates.
Such mechanisms rely on interpretive conventions supplied by extension layers
and therefore permit divergent admissibility judgments under persistent
disagreement.

\subsection{Optimization Criterion}

The optimization objective of this paper is to minimize explicit ontological
structure subject to the constraints above.
Minimality is not defined as 
the smallest possible number of primitives in the abstract, 
but as the smallest number of explicit identity-and-persistence
regimes required to preserve neutrality and stable reference.

A proposed reduction is counted as a simplification only if it does not
reintroduce the same distinctions through hidden or interpretive mechanisms.
Collapsing regimes while reintroducing their distinctions via internal
discriminators, contextual typing, or role predicates does not reduce total
structural complexity and is therefore excluded by the optimization criterion
adopted here.

Alternative optimization objectives prioritize different tradeoffs, such as
interpretive coordination, domain-specific efficiency, or context-dependent
classification at the substrate level.
The results presented here apply only to substrates optimized for
interoperability under persistent disagreement.

\subsection{Implications for the Results}

Under the assumptions stated, the central question is not which ontology is
most expressive or convenient, but which structural distinctions are
unavoidable.
The following sections formalize the representational requirements imposed by
accountability analysis and derive lower-bound constraints on ontological
structure.
These constraints are independent of domain, policy area, or analytic method,
and depend only on the optimization objective fixed here.
