% !TeX root = se200_identity_regimes.tex
\section{Conclusion}
\label{sec:conclusion}

This paper has addressed a structural question that arises when ontologies
are required to function as neutral substrates under persistent
disagreement.
Under explicit constraints of neutrality, stability under incompatible extension, 
and the exclusion of hidden or interpretive distinctions, we have
shown that a lower bound exists on the ontological structure 
required to support accountability.
Ontologies with fewer than six distinct identity-and-persistence regimes
cannot satisfy the stated neutrality and stability requirements;
a construction with exactly six regimes suffices without excess structure.

The result is conditional rather than prescriptive.
It does not claim that the identified structure 
is optimal for all applications, 
nor does it assess the adequacy of 
alternative modeling strategies for other objectives.
It establishes only that, under a particular and explicitly stated
optimization objective, certain distinctions are structurally unavoidable.

This work clarifies the design space for neutral substrates under persistent
disagreement.
Under the stated constraints, it shows that minimality is fixed by structural
requirements rather than by modeling preference.
Future work may examine how different optimization objectives yield
alternative bounds, or assess 
how the six-regime structure performs in
practical accountability applications.

\section*{Acknowledgements}

Portions of this work were developed
using computer-assisted tools during manuscript preparation.
Generative language models were used to assist with editing, formatting,
and consistency checking during manuscript preparation.
All conceptual framing, formal development, results, interpretations, and conclusions
are the author's own.
All generated suggestions were critically reviewed and validated, and
the author takes full responsibility for the content of this work.
