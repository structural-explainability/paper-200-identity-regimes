% !TeX root = se200_identity_regimes.tex
\section{Necessity: A Lower Bound on Ontological Structure}
\label{sec:necessity}

This section establishes a lower bound on ontological structure under the
assumptions and optimization criterion fixed in Section~\ref{sec:assumptions}.
The result is a necessity claim: if a substrate is to support accountability
under persistent disagreement while maintaining neutrality and stable
reference, then a minimum number of distinct identity-and-persistence regimes
is unavoidable.

The argument proceeds in two steps.
First, we identify the representational capacities that an accountability
substrate must support, independent of any particular ontology.
Second, we show that collapsing any two of the resulting identity regimes
either violates stable reference or reintroduces hidden distinctions that are
inadmissible under the stated optimization.

\subsection{Identity-and-Persistence Regimes}

Under the neutrality constraint, ontological distinctions cannot be grounded
in function, role, or interpretation.
They must instead be grounded in differences in identity, persistence, and
admissible relations.
An \emph{identity-and-persistence regime} specifies what it means for an
entity to remain the same entity over time and across incompatible extensions,
and which relations it may admit without reclassification.

Accountability analysis requires support for at least the following distinct
regimes:

\begin{enumerate}
  \item \textbf{Obligation-bearing entities:} enduring referents that may bear
        rights, obligations, or responsibilities and remain identifiable as
        parties across time.
  \item \textbf{Authority-bearing structures:} enduring referents that ground
        obligations or permissions without being tied to any single
        occurrence.
  \item \textbf{Acted-upon referents:} enduring non-obligation-bearing entities
        that may be regulated, operated on, or affected by occurrences.
  \item \textbf{Time-indexed occurrences:} entities individuated by their
        temporal realization and provenance, recording that something
        happened.
  \item \textbf{Applicability contexts:} referents that scope where, to whom,
        or under what conditions authority applies, and which may be nested or
        overlapping.
  \item \textbf{Descriptive records or indicators:} referents that report
        measured or derived properties without asserting causal or normative
        conclusions.
\end{enumerate}

Each regime differs in its identity conditions, persistence behavior, and
admissible relations.
Treating these differences as merely contextual or role-based distinctions
would make admissibility depend on interpretation, violating the requirement
of stable reference.

\subsection{Collapse Strategy and Admissibility}

To establish a lower bound, we consider whether any two of the regimes above
can be collapsed into a single identity regime without violating the assumptions of
Section~\ref{sec:assumptions}.
A collapse is admissible only if the resulting representation admits a single,
extension-stable identity regime and does not require hidden discriminators to
recover the original distinction.

For each possible collapse, at least one of the following outcomes arises:

\begin{itemize}
  \item \textbf{Identity instability:} the collapsed regime requires
        incompatible persistence criteria, so that identity varies across
        extensions.
  \item \textbf{Category error:} the collapsed regime admits relations that are
        structurally inappropriate, requiring interpretive rules to block
        them.
  \item \textbf{Hidden regime reintroduction:} the distinction is reintroduced
        via roles, flags, or contextual predicates, violating the
        no-smuggling criterion.
\end{itemize}

Because such mechanisms rely on interpretive agreement supplied by extension
layers, they are excluded by the stated optimization objective.

\subsection{Lower-Bound Result}

\begin{lemma}[Necessity of distinct regimes]
\label{lem:necessity}
Under the assumptions of Section~\ref{sec:assumptions}, no ontology whose stable partition
contains fewer than six distinct identity-and-persistence regimes can satisfy
the representational requirements of accountability while remaining neutral
and stable under incompatible extension.
\end{lemma}

\begin{proof}[Proof sketch]
Suppose an ontology realizes fewer than six identity-and-persistence regimes.
Then at least two of the regimes required for accountability are collapsed
into a single identity regime.
By the collapse analysis above, any such collapse either forces incompatible
identity criteria, admits structurally invalid relations, or requires internal
discriminators whose interpretation is supplied by extension layers.
Each case violates either stable reference or the neutrality constraint.
Therefore no ontology with fewer than six regimes can satisfy all stated
requirements simultaneously.
\end{proof}

\subsection{Interpretation of the Necessity Result}

The necessity result is conditional and structural.
It does not claim that fewer regimes are impossible in general, but that they
are incompatible with the optimization objective fixed in Section~\ref{sec:assumptions}.
Ontologies optimized for local agreement, negotiated semantics, or expressive
flexibility may legitimately employ fewer explicit distinctions by relying on
hidden or interpretive mechanisms.

Within the stated constraints, however, the result establishes a sharp lower
bound: any neutral substrate capable of supporting accountability under
persistent disagreement must realize at least six distinct
identity-and-persistence regimes.
The following section shows that this bound is tight by exhibiting a
construction that satisfies all requirements with a minimal set of six regimes.
