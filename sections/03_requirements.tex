% !TeX root = se200_identity_regimes.tex
\section{Structural Requirements for Accountability Substrates}
\label{sec:requirements}

Given the assumptions and optimization criterion fixed in the previous
section, we now state the representational requirements that an ontological
substrate must satisfy in order to support accountability analysis under
persistent disagreement.
These requirements are structural rather than interpretive.
They do not prescribe how accountability should be understood, evaluated, or
enforced; instead, they specify what must be representable in a stable and
neutral manner.

The requirements are framed as necessary capacities.
They are not derived from any particular domain, policy area, or theory of
governance, but from the minimal conditions under which accountability claims
can be stated, inspected, and compared without embedding causal or normative
commitments in the substrate.

\subsection{R1: Stable Identity and Persistence}

The substrate must support stable reference to entities over time.
Each entity must have identity and persistence conditions that are invariant
under incompatible extensions.
Changes in interpretation, role, status, or evaluation must not require
reclassification of entities or modification of their identity criteria.

This requirement does not apply to representations in which identity is
determined contextually or retroactively by interpretive layers.
If an entity must change its ontological status in order to participate in new
relationships or analyses, stable reference is not preserved.

\subsection{R2: Obligation-Bearing Capacity}

The substrate must support representation of entities that can bear rights,
obligations, or responsibilities.
Such entities must be referable independently of the particular occurrences
in which obligations are fulfilled, violated, or discharged.

This requirement does not presuppose any substantive theory of normativity.
It requires only that obligation-bearing entities be representable as enduring
referents whose identity does not depend on specific events.

\subsection{R3: Normative Reference Without Execution}

The substrate must distinguish between the existence of authority or
obligation and its execution.
Normative or regulatory structures must therefore be representable
independently of the occurrences that enact, comply with, or violate them.

This distinction is necessary to support accountability over time, including
cases of delayed enforcement, partial compliance, amendment, or contested application.
Encoding normative reference solely through occurrences collapses authority
into execution and undermines stable reference to governing structures.

\subsection{R4: Time-Indexed Occurrence}

The substrate must support representation of time-indexed occurrences.
Occurrences must be individuated by their temporal realization and provenance,
and must be distinguishable from enduring entities.

This requirement applies to actions as well as to other discrete happenings
relevant to accountability, such as filings, inspections, or transactions.
Occurrences are not required to carry interpretive or evaluative meaning; they
must only assert that something occurred at a particular time and place.

\subsection{R5: Applicability and Scope}

The substrate must support representation of applicability contexts.
It must be possible to state where, to whom, or under what conditions a
normative or regulatory structure applies, without collapsing such contexts
into either obligation-bearing entities or physical loci.

Applicability contexts may be nested, overlapping, or partially coincident.
They must therefore be representable as stable referents in their own right,
rather than as incidental attributes whose interpretation may vary across
extensions.

\subsection{R6: Descriptive Indicators Without Causal Commitment}

The substrate must support representation of descriptive indicators or
records that characterize states, performance, or outcomes associated with
entities or occurrences.
Such indicators must be representable without asserting causal relations,
evaluative judgments, or explanatory claims.

This requirement is necessary to support longitudinal comparison and outcome
inspection while preserving neutrality.
Collapsing indicators into occurrences or into interpretive annotations
eliminates stable reference to descriptive records across time and analytic
frameworks.

\subsection{Summary}

Together, R1--R6 specify the minimal representational capacities required for
an accountability substrate optimized for interoperability under persistent
disagreement.
They do not yet impose a particular ontological partition.
Rather, they define the constraints under which any such partition must
operate.

In the following section, these requirements are used to derive a lower bound
on the number of distinct identity-and-persistence regimes that a substrate
must realize in order to satisfy them simultaneously.
