% !TeX root = se200_identity_regimes.tex
\section{Related Work}
\label{sec:related}

This paper is situated at the intersection of formal ontology, knowledge
representation, and the study of classification under disagreement.
Rather than surveying ontology engineering broadly, it situates the
present result with respect to prior work on neutrality, identity, and
structural stability.

\subsection{Classification, Disagreement, and Neutrality}

A substantial body of work has shown that classification systems embed
institutional, social, and political commitments rather than functioning
as neutral descriptive devices~\cite{bowker1999sorting,haslanger2012resisting}.
In scientific and policy contexts, disagreement is often persistent rather
than transient, reflecting incompatible evaluative frameworks rather than
resolvable empirical uncertainty~\cite{longino1990science}.

These observations motivate the need for representations that remain usable
without assuming interpretive convergence.
The present work adopts this motivation but addresses a distinct question:
what minimal structural commitments are unavoidable when disagreement is treated as 
permanent and interpretive commitments are externalized rather than negotiated?

\subsection{Identity, Roles, and Ontological Methodology}

Formal ontology has long emphasized the role of identity conditions and
persistence criteria in category formation.
The OntoClean methodology~\cite{guarino2002evaluating} clarifies why roles are
anti-rigid and unsuitable as identity-defining categories~\cite{guarino1998roles},
and subsequent work on social and institutional entities has further
developed these distinctions~\cite{masolo2004wonderweb}.

This paper builds on these insights but applies them at the substrate level.
The present analysis operates at a different level than role-based modeling, 
deriving substrate-level identity-and-persistence regimes 
from neutrality requirements rather than from domain classifications.

\subsection{Upper Ontologies}

Upper ontologies such as BFO~\cite{smith2015basic}, DOLCE~\cite{gangemi2002dolce},
and UFO~\cite{guizzardi2005ontological} provide rich accounts of dependence,
endurants and perdurants, and social objects.
They differ substantially in commitments and optimization objectives, and
are not designed primarily as neutral substrates for interoperability under
persistent disagreement.

The present work does not compare expressivity or advocate alignment with any
particular upper ontology.
Instead, it derives lower-bound constraints that apply independently of such
frameworks, conditional on the neutrality and stability requirements adopted
here.
Whether the regimes derived here 
align with, refine, or cross-cut the category structures of 
existing upper ontologies is an open question 
beyond the scope of this paper.

\subsection{Relation to Common Logic and ontology exchange}

The neutral substrate and derived regimes can be
expressed in ISO/IEC 24707 Common Logic as a conservative theory, 
supporting interpreter-independent exchange 
without reliance on description-logic restrictions or fixed signatures.
Here, conservative means that 
extensions add structure 
without altering the consequences 
derivable about substrate-level terms, 
ensuring that incompatible interpretive frameworks 
do not retroactively change the meaning of shared references.
This enables compatibility with ontology exchange workflows 
widely used in applied ontology and digital twin contexts.

\subsection{Causality, Explanation, and Infrastructure}

Separating descriptive structure from causal or explanatory claims is a
well-established principle in causal analysis~\cite{pearl2009causality}.
Similarly, research on large-scale information infrastructures emphasizes
the importance of long-term stability and interpretive restraint across
institutional boundaries~\cite{edwards2011infrastructure}.

These perspectives align with the pre-causal and pre-normative constraints assumed in this paper.
The ontology analyzed here is not an explanatory theory, but an
infrastructural substrate intended to support explanation, evaluation, and
policy analysis without embedding them.

\subsubsection{Relation to FAIR Data Principles}

The FAIR principles promote data reuse across communities and over time,
but do not specify the ontological constraints required for interoperability
under persistent disagreement.
In practice, reuse is often pursued through 
negotiated vocabularies or shared interpretations, 
which presuppose forms of convergence that may not be
available in legal, political, or analytic settings.

The results of this paper clarify a structural precondition for FAIR-aligned
reuse in pluralistic or institutionally heterogeneous contexts:
stable reference must be supported independently of 
causal, normative, or evaluative agreement.
The identity-and-persistence regimes derived here therefore constrain the
ontological substrates upon which FAIR-compliant systems implicitly rely
when reuse is expected across incompatible extensions.
