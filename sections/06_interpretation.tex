% !TeX root = se200_identity_regimes.tex
\section{Interpretation of the Result}
\label{sec:interpretation}

This section clarifies how the lower- and upper-bound results should be
understood, and how they should not be understood.
The aim is not to defend the result against alternative modeling goals,
but to situate it relative to the explicit optimization criterion fixed
earlier in the paper.

\subsection{What the Result Establishes}

Taken together, the necessity and sufficiency results establish a
conditional optimality claim.
Under the assumptions of Section~\ref{sec:assumptions}, 
any ontology intended to function as a neutral substrate 
for accountability under persistent disagreement
must realize at least six distinct identity-and-persistence regimes.
Conversely, a construction with a minimal set of six such regimes suffices to meet
all stated representational requirements without introducing hidden
structure or interpretive commitments.

The result is therefore tight.
It does not merely show that six regimes are adequate, 
nor that fewer regimes are inadequate under particular conditions, 
but that six regimes are both necessary and sufficient 
under the stated optimization objective.

\subsection{Minimality Relative to Stability}

The result should not be read as a claim about simplicity in an intuitive
or engineering sense.
Ontologies with fewer explicit entity kinds may be easier to describe or extend
within a coordinated interpretive community.
However, when such reductions are achieved by encoding stability-critical
distinctions as roles, flags, or contextual predicates, total structural
complexity is not reduced.
Instead, it is displaced into extension layers whose interpretation cannot
be held fixed under disagreement.

In this paper, minimality is defined relative to stability.
A representation is minimal only if removing explicit structure does not
require reintroducing the same distinctions implicitly.
From this perspective, an ontology with fewer explicit kinds but greater
interpretive burden is not simpler, but less explicit.

\subsection{Why Hidden Regimes Are Excluded}

Hidden regimes are often effective where interpretation is shared,
enforced, or inexpensive to renegotiate.
They support flexibility and expressive economy within a single
governance framework.

The present analysis excludes such regimes not because they are incorrect,
but because they are incompatible with the stated objective.
When disagreement is persistent and uncoordinated, distinctions that are
not fixed at the substrate level cannot be relied upon to preserve identity
or admissible relations.
Stability then depends on interpretive convergence, which the substrate is
explicitly designed not to assume.

\subsection{Relation to Other Ontological Strategies}

Many upper ontologies and modeling frameworks adopt different optimization
objectives, including metaphysical coverage, expressive richness, or
alignment with natural language or scientific theory.
Such frameworks may introduce additional kinds or tolerate contextual
typing precisely in order to support richer modeling.

The result presented here neither subsumes nor contradicts those approaches.
It identifies a specific point in the design space: the minimal structure
required for a neutral, extension-stable substrate under persistent
disagreement.
Frameworks optimized for other goals may legitimately occupy different
points in that space.

\subsection{Conditional Scope of the Claim}

The lower-bound result is explicitly conditional.
If any assumption of Section~\ref{sec:assumptions} is rejected, 
such as neutrality, stability under incompatible extension, 
or the exclusion of hidden regimes, 
the optimization criterion is invalidated 
and the bound no longer applies.

Within the stated scope, however, the result is unavoidable.
Any attempt to reduce the number of regimes must either sacrifice stability,
reintroduce hidden distinctions, or weaken the neutrality constraint.
The six-regime structure thus emerges not as a design preference, but as a
structural consequence of the problem as posed.

\subsection{Implications}

The primary implication of this result is diagnostic rather than
prescriptive.
It provides a principled way to evaluate claims that an ontology supports
neutral interoperability under disagreement.
If such a framework relies on implicit roles or context-dependent typing
for stability-critical distinctions, the present result explains why those
claims cannot be sustained simultaneously under the stated constraints.

Conversely, for designers who share the stated optimization objective, the
result identifies which distinctions must be made explicit, and which may
safely be deferred to extension layers, in order to preserve neutrality and
stable reference across incompatible interpretations.
