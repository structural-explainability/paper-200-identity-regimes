\PassOptionsToPackage{colorlinks=true,allcolors=black}{hyperref}
% REQ-FILE: The above line, \PassOptionsToPackage{} must be very first line.

\documentclass[11pt]{article}

% ============================================================
% Annotation visibility toggle
% Build annotated PDF by defining \ANNOTATED (e.g., via latexmk flag).
% Requires xcolor loaded first, then use \textcolor and \color commands.
% ============================================================
\newif\ifShowAnnotations
\ifdefined\ANNOTATED
  \ShowAnnotationstrue
\else
  \ShowAnnotationsfalse
\fi

% Encoding and font setup for cross-platform reproducibility
\usepackage[T1]{fontenc}
\usepackage{lmodern}

% Improve readability for dense theoretical material
\usepackage{setspace}
\onehalfspacing

% Mathematical notation and table formatting for formal semantics
\usepackage{amsmath, amssymb}
\usepackage{array}
\usepackage[table]{xcolor}
\usepackage{colortbl}
\usepackage{booktabs}
\usepackage{graphicx}

% Theorem environments (best-practice, widely standard)
\usepackage{amsthm}

% ------------------------------------------------------------
% Numbering convention:
% - All theorem-like statements share one counter, numbered by section.
% - Definition/Example/Remark/Note also share the same counter family.
%   (This is the most common convention in math/logic/category theory writing.)
% ------------------------------------------------------------

% Italic body text: results
\theoremstyle{plain}
\newtheorem{theorem}{Theorem}[section]
\newtheorem{lemma}[theorem]{Lemma}
\newtheorem{proposition}[theorem]{Proposition}
\newtheorem{corollary}[theorem]{Corollary}
\newtheorem{conjecture}[theorem]{Conjecture}

% Claim varies; most standard is to share the theorem counter
\newtheorem{claim}[theorem]{Claim}

% Upright body text: definitions and examples
\theoremstyle{definition}
\newtheorem{definition}[theorem]{Definition}
\newtheorem{example}[theorem]{Example}

% Upright body text: remarks/notes
\theoremstyle{remark}
\newtheorem{remark}[theorem]{Remark}
\newtheorem{note}[theorem]{Note}

% Optional: unnumbered versions for front-matter previews/callouts
% (This is also standard when you want a preview without numbering.)
\newtheorem*{theorem*}{Theorem}
\newtheorem*{lemma*}{Lemma}
\newtheorem*{proposition*}{Proposition}
\newtheorem*{corollary*}{Corollary}
\newtheorem*{conjecture*}{Conjecture}
\newtheorem*{claim*}{Claim}
\newtheorem*{definition*}{Definition}
\newtheorem*{example*}{Example}
\newtheorem*{remark*}{Remark}
\newtheorem*{note*}{Note}

% Bibliography management and hyperlink support
\usepackage{url}
\usepackage{natbib}
% load hyperref before bookmark to avoid warnings
\usepackage{hyperref}
\usepackage{bookmark}

% Support multiple author affiliations
\usepackage{authblk}

% Improve typographic quality and line breaking
\usepackage{microtype}

% Categorical and commutative diagrams
\usepackage{tikz}
\usepackage{tikz-cd}
\usetikzlibrary{arrows.meta,calc,fit,positioning,shapes.geometric}

% Callout boxes for examples and emphasis
\usepackage[most]{tcolorbox}

% Keywords macro for structured abstract metadata
\providecommand{\keywords}[1]{\textbf{Keywords: } #1}

% Notation for Raw vs Canonical categories in EP/CEE
\newcommand{\Raw}{\mathsf{Raw}}
\newcommand{\Canon}{\mathsf{Canon}}

% Reusable figure callout box for visual emphasis
\newcommand{\FigureCallout}[2]{%
  \begin{tcolorbox}[
    colback=gray!5,
    colframe=black!40,
    title={#1},
    fonttitle=\bfseries,
    arc=3pt,
    boxrule=0.5pt,
    width=\linewidth,
    enhanced,
    breakable
  ]
  #2
  \end{tcolorbox}
}
% Notation for stability-critical 
\newcommand{\stabilitycritical}{stability-critical}

% ============================================================
% Annotations (REQ / WHY / OBS) scan-friendly
% ============================================================
\newcommand{\AnnColor}{gray} % xcolor color name
\newlength{\AnnHangIndent}
\setlength{\AnnHangIndent}{1.8em} % hanging indent amount

% Single annotation line/block:
% - small gray text
% - bold label REQ(P0) then content
% - hanging indent for scanning
\newcommand{\AnnInline}[3]{%
  \ifShowAnnotations
    \par\begingroup
      \footnotesize
      \color{\AnnColor}%
      \noindent
      \hangindent=\AnnHangIndent
      \hangafter=1
      \textbf{#1(#2):}~#3\par
    \endgroup
    \vspace{0.15\baselineskip}%
  \fi
}
\newcommand{\REQ}[2]{\AnnInline{REQ}{#1}{#2}}
\newcommand{\WHY}[2]{\AnnInline{WHY}{#1}{#2}}
\newcommand{\OBS}[2]{\AnnInline{OBS}{#1}{#2}}

% Paragraph spacing prioritizes readability over traditional indentation
\setlength{\parskip}{0.75em}
\setlength{\parindent}{0em}

% Compact list formatting for dense conceptual content
\usepackage{enumitem}
\setlist[itemize]{itemsep=0.2em, topsep=0.2em, parsep=0em, partopsep=0em}

% Defensive re-definition ensures commands exist if loaded earlier
\providecommand{\Raw}{\mathrm{Raw}}
\providecommand{\Canon}{\mathrm{Canon}}

% ============================================================
\title{Substrate Stability Under Persistent Disagreement: \\
Structural Constraints for \\ Neutral Ontological Substrates }
% ============================================================

\author[1,2]{Denise M. Case}
\affil[1]{Northwest Missouri State University, Computer Science and Information Systems, Maryville, MO, USA}
\affil[2]{Civic Interconnect, Ely, MN, USA}

\date{\today}

\begin{document}

\maketitle
\vspace{-1em}

% Explicitly mark preprint status
\begin{tcolorbox}[colback=gray!5, colframe=black!20, boxrule=0.3pt]
  \textbf{Preprint Notice.}
  This preprint has not undergone peer review.
  It is shared to support community discussion, transparency research, and early technical evaluation.
\end{tcolorbox}

\begin{abstract}
Modern data systems increasingly operate under conditions of persistent legal,
political, and analytic disagreement.
In such settings, interoperability cannot rely on shared interpretation,
negotiated semantics, or centralized authority.
Instead, representations must function as neutral substrates that preserve
stable reference across incompatible extensions.

This paper investigates the structural constraints imposed on ontological
design by this requirement.
Building on a neutrality framework that treats interpretive non-commitment and
stability under extension as explicit design constraints, we ask what minimal
ontological structure is forced if accountability relationships are to remain
referable and comparable under disagreement.
Minimality here is not mere parsimony:
a reduction is admissible only if it does not 
reintroduce \stabilitycritical{} distinctions 
as hidden roles, flags, or contextual predicates.

We establish a conditional lower-bound result: any ontology capable of
supporting accountability under persistent disagreement must realize at least
six distinct identity-and-persistence regimes.
We further show that a construction with exactly six such regimes is sufficient
to satisfy the stated requirements without embedding causal or normative
commitments in the substrate.
The result is not a proposal for a universal ontology, but a constraint on what
is possible when neutrality and stable reference are treated as non-negotiable
design goals.
\end{abstract}


% =============================================================================
% Theorem Preview 
% =============================================================================

\bigskip
\noindent\textbf{Statement of Result (Preview).}
\emph{Under explicit neutrality and stability constraints, we show that attempts to
simplify ontological structure by collapsing distinctions inevitably reintroduce
those distinctions as hidden interpretive mechanisms.
When such hidden regimes are excluded, accountability under persistent
disagreement requires a minimal set of six distinct identity-and-persistence regimes:
fewer do not satisfy the stated neutrality and stability requirements, 
and more are unnecessary.}

\medskip
\noindent
Informally: Neutrality is achieved by
externalizing explanatory and evaluative structure.
To function as a substrate for disagreement, an ontology must remain
agnostic with respect to causal and normative interpretation.

\medskip
\noindent
The formal statement and proof appear in Section~\ref{sec:interpretation}.

\keywords{
  formal ontology;
  ontological neutrality;
  accountability;
  neutral substrates;
  extension stability;
  causal and normative commitment
}

\ifShowAnnotations
  \clearpage
  \input{sections/00_contract}
  \clearpage
\fi

% !TeX root = se200_identity_regimes.tex
\section{Introduction: Structure Under Disagreement}
\label{sec:intro}

Modern data systems increasingly operate in environments characterized by
persistent legal, political, and analytic disagreement.
In such settings, no single interpretive authority can be assumed, and
semantic convergence cannot be enforced without excluding legitimate
participants or perspectives.
Nevertheless, shared use of data remains necessary:
entities must be referable, actions traceable, and
records comparable across frameworks that may diverge
in assumptions, methods, or evaluative standards.

This paper addresses a structural question that arises in this context:
\emph{what ontological structure is required if a representation is to remain
usable as a shared substrate across incompatible interpretations?}
The focus is not on domain modeling, explanatory adequacy, or normative correctness.
Rather, it concerns the structural conditions required for stable reference
and accountability under persistent disagreement.

We adopt a neutrality framework developed in prior work~\cite{case2025neutrality},
treating ontological neutrality as a design constraint rather than an
aspirational ideal.
Neutrality is understood here as interpretive non-commitment combined with
stability under incompatible extension.
Under this view, a substrate must permit independent extensions to introduce
causal, normative, or analytic structure without requiring revision of the
underlying representation.
Accordingly, explanatory and evaluative structure is externalized, and the
substrate is constrained to remain pre-causal and pre-normative at the
substrate level.

Within this setting, ontological minimality cannot be understood solely as a
reduction in the number of primitives.
Many apparent simplifications achieve parsimony by reintroducing essential
distinctions indirectly, through roles, flags, contextual typing, or other
internal discriminators whose interpretation is supplied by extension layers
(e.g., a single undifferentiated entity type distinguished only by role
predicates such as \emph{isAgent}, \emph{isOccurrence}, or \emph{isAuthority}).
Such encodings rely on shared interpretive assumptions, which may not be
available in settings characterized by persistent disagreement.
For this reason, the present analysis excludes reductions that preserve
stability only by deferring essential distinctions to interpretive mechanisms
not represented at the substrate level.

The specific focus of this paper is accountability.
Accountability analysis requires stable reference to participants, authority
structures, regulated occurrences, applicability contexts, and descriptive
records, without embedding causal claims or normative judgments in the
substrate itself.
These categories are not introduced as an ontological partition,
but as representational capacities whose simultaneous support
places structural constraints on any neutral substrate.
The question addressed here is not how these elements should be interpreted,
but whether a neutral substrate can support them at all, and if so, with what
irreducible structural commitments.

The central contribution of this paper is a conditional bound on ontological
structure given the design goals outlined above.
Under explicit neutrality and stability constraints, we show that any ontology
capable of supporting accountability under persistent disagreement must
realize at least six distinct identity-and-persistence regimes.
We further show that a construction with a minimal set of six such regimes is
sufficient to satisfy the stated requirements without introducing hidden or
implicit structure.
The result is not a proposal for a universal ontology, but a constraint on
what is possible under a specific optimization objective.

Alternative optimization objectives prioritize different tradeoffs, such as
interpretive convergence, domain-specific efficiency, or context-dependent
classification at the substrate level.
The results presented here apply only under the stated constraints, within
which a minimal, domain-independent bound on ontological structure is
established for stable reference and accountability.

The remainder of the paper proceeds as follows.
Section~\ref{sec:assumptions} states the assumptions, neutrality constraints,
and optimization criterion.
Section~\ref{sec:requirements} formalizes the structural requirements imposed
by accountability analysis.
Section~\ref{sec:necessity} derives a lower bound on the number of required
identity-and-persistence regimes.
Section~\ref{sec:sufficiency} presents a construction achieving this bound.
Section~\ref{sec:interpretation} discusses the scope and implications of the
result.
Section~\ref{sec:related} situates the work relative to existing ontology
frameworks.
Section~\ref{sec:conclusion} concludes.
           
% !TeX root = se200_identity_regimes.tex
\section{Assumptions, Neutrality, and Optimization Criterion}
\label{sec:assumptions}

This section fixes the problem setting for the remainder of the paper.
All results that follow are conditional on the assumptions stated here.
The purpose is not to argue that these assumptions are universally correct,
but to make explicit the constraints under which a neutral substrate for
accountability is possible.

\subsection{Persistent Disagreement and Uncoordinated Extension}

We assume environments in which disagreement is not an anomaly to be
resolved, but a persistent condition.
Legal interpretation, policy evaluation, and analytic methodology may remain
incompatible across communities and over time.
Accordingly, no single interpretive authority or shared semantic framework
can be assumed at the substrate level.

We further assume that extensions to the substrate may occur without
coordination.
Independent actors may add explanatory, normative, or analytic structure,
and such extensions must not require revision of the substrate itself.
This assumption does not rely on 
global semantic agreement, centralized governance, or 
negotiated meaning as preconditions for interoperability.

\subsection{Neutrality Constraint}

This work assumes substrate neutrality and stability 
as a design constraint~\cite{case2025neutrality} 
rather than as a philosophical claim.
A neutral substrate does not assert why
events occur, whether outcomes are desirable, or 
how actions ought to be evaluated.
Instead, it provides stable referents and structural relations upon which
external interpretive layers may operate.

Neutrality further requires stability under incompatible extension.
If two extensions disagree in interpretation, evaluation, or explanation,
their disagreement must not force reclassification of entities, alteration
of identity conditions, or revision of the substrate.
Any representation whose admissibility conditions depend on interpretive
agreement does not satisfy this constraint.

\subsection{Stable Reference Requirement}

For a substrate to support accountability, entities must remain referable
over time and across extensions.
Identity conditions and admissible relationships must be invariant under
changes in interpretation.
An entity may participate in new relationships or acquire additional
annotations, but it must not change its ontological status in order to remain
usable as a shared referent.

This requirement excludes representations in which stability-critical
distinctions are encoded only as roles, flags, or contextual predicates.
Such mechanisms rely on interpretive conventions supplied by extension layers
and therefore permit divergent admissibility judgments under persistent
disagreement.

\subsection{Optimization Criterion}

The optimization objective of this paper is to minimize explicit ontological
structure subject to the constraints above.
Minimality is not defined as 
the smallest possible number of primitives in the abstract, 
but as the smallest number of explicit identity-and-persistence
regimes required to preserve neutrality and stable reference.

A proposed reduction is counted as a simplification only if it does not
reintroduce the same distinctions through hidden or interpretive mechanisms.
Collapsing regimes while reintroducing their distinctions via internal
discriminators, contextual typing, or role predicates does not reduce total
structural complexity and is therefore excluded by the optimization criterion
adopted here.

Alternative optimization objectives prioritize different tradeoffs, such as
interpretive coordination, domain-specific efficiency, or context-dependent
classification at the substrate level.
The results presented here apply only to substrates optimized for
interoperability under persistent disagreement.

\subsection{Implications for the Results}

Under the assumptions stated, the central question is not which ontology is
most expressive or convenient, but which structural distinctions are
unavoidable.
The following sections formalize the representational requirements imposed by
accountability analysis and derive lower-bound constraints on ontological
structure.
These constraints are independent of domain, policy area, or analytic method,
and depend only on the optimization objective fixed here.
        
% !TeX root = se200_identity_regimes.tex
\section{Structural Requirements for Accountability Substrates}
\label{sec:requirements}

Given the assumptions and optimization criterion fixed in the previous
section, we now state the representational requirements that an ontological
substrate must satisfy in order to support accountability analysis under
persistent disagreement.
These requirements are structural rather than interpretive.
They do not prescribe how accountability should be understood, evaluated, or
enforced; instead, they specify what must be representable in a stable and
neutral manner.

The requirements are framed as necessary capacities.
They are not derived from any particular domain, policy area, or theory of
governance, but from the minimal conditions under which accountability claims
can be stated, inspected, and compared without embedding causal or normative
commitments in the substrate.

\subsection{R1: Stable Identity and Persistence}

The substrate must support stable reference to entities over time.
Each entity must have identity and persistence conditions that are invariant
under incompatible extensions.
Changes in interpretation, role, status, or evaluation must not require
reclassification of entities or modification of their identity criteria.

This requirement does not apply to representations in which identity is
determined contextually or retroactively by interpretive layers.
If an entity must change its ontological status in order to participate in new
relationships or analyses, stable reference is not preserved.

\subsection{R2: Obligation-Bearing Capacity}

The substrate must support representation of entities that can bear rights,
obligations, or responsibilities.
Such entities must be referable independently of the particular occurrences
in which obligations are fulfilled, violated, or discharged.

This requirement does not presuppose any substantive theory of normativity.
It requires only that obligation-bearing entities be representable as enduring
referents whose identity does not depend on specific events.

\subsection{R3: Normative Reference Without Execution}

The substrate must distinguish between the existence of authority or
obligation and its execution.
Normative or regulatory structures must therefore be representable
independently of the occurrences that enact, comply with, or violate them.

This distinction is necessary to support accountability over time, including
cases of delayed enforcement, partial compliance, amendment, or contested application.
Encoding normative reference solely through occurrences collapses authority
into execution and undermines stable reference to governing structures.

\subsection{R4: Time-Indexed Occurrence}

The substrate must support representation of time-indexed occurrences.
Occurrences must be individuated by their temporal realization and provenance,
and must be distinguishable from enduring entities.

This requirement applies to actions as well as to other discrete happenings
relevant to accountability, such as filings, inspections, or transactions.
Occurrences are not required to carry interpretive or evaluative meaning; they
must only assert that something occurred at a particular time and place.

\subsection{R5: Applicability and Scope}

The substrate must support representation of applicability contexts.
It must be possible to state where, to whom, or under what conditions a
normative or regulatory structure applies, without collapsing such contexts
into either obligation-bearing entities or physical loci.

Applicability contexts may be nested, overlapping, or partially coincident.
They must therefore be representable as stable referents in their own right,
rather than as incidental attributes whose interpretation may vary across
extensions.

\subsection{R6: Descriptive Indicators Without Causal Commitment}

The substrate must support representation of descriptive indicators or
records that characterize states, performance, or outcomes associated with
entities or occurrences.
Such indicators must be representable without asserting causal relations,
evaluative judgments, or explanatory claims.

This requirement is necessary to support longitudinal comparison and outcome
inspection while preserving neutrality.
Collapsing indicators into occurrences or into interpretive annotations
eliminates stable reference to descriptive records across time and analytic
frameworks.

\subsection{Summary}

Together, R1--R6 specify the minimal representational capacities required for
an accountability substrate optimized for interoperability under persistent
disagreement.
They do not yet impose a particular ontological partition.
Rather, they define the constraints under which any such partition must
operate.

In the following section, these requirements are used to derive a lower bound
on the number of distinct identity-and-persistence regimes that a substrate
must realize in order to satisfy them simultaneously.
    
% !TeX root = se200_identity_regimes.tex
\section{Necessity: A Lower Bound on Ontological Structure}
\label{sec:necessity}

This section establishes a lower bound on ontological structure under the
assumptions and optimization criterion fixed in Section~\ref{sec:assumptions}.
The result is a necessity claim: if a substrate is to support accountability
under persistent disagreement while maintaining neutrality and stable
reference, then a minimum number of distinct identity-and-persistence regimes
is unavoidable.

The argument proceeds in two steps.
First, we identify the representational capacities that an accountability
substrate must support, independent of any particular ontology.
Second, we show that collapsing any two of the resulting identity regimes
either violates stable reference or reintroduces hidden distinctions that are
inadmissible under the stated optimization.

\subsection{Identity-and-Persistence Regimes}

Under the neutrality constraint, ontological distinctions cannot be grounded
in function, role, or interpretation.
They must instead be grounded in differences in identity, persistence, and
admissible relations.
An \emph{identity-and-persistence regime} specifies what it means for an
entity to remain the same entity over time and across incompatible extensions,
and which relations it may admit without reclassification.

Accountability analysis requires support for at least the following distinct
regimes:

\begin{enumerate}
  \item \textbf{Obligation-bearing entities:} enduring referents that may bear
        rights, obligations, or responsibilities and remain identifiable as
        parties across time.
  \item \textbf{Authority-bearing structures:} enduring referents that ground
        obligations or permissions without being tied to any single
        occurrence.
  \item \textbf{Acted-upon referents:} enduring non-obligation-bearing entities
        that may be regulated, operated on, or affected by occurrences.
  \item \textbf{Time-indexed occurrences:} entities individuated by their
        temporal realization and provenance, recording that something
        happened.
  \item \textbf{Applicability contexts:} referents that scope where, to whom,
        or under what conditions authority applies, and which may be nested or
        overlapping.
  \item \textbf{Descriptive records or indicators:} referents that report
        measured or derived properties without asserting causal or normative
        conclusions.
\end{enumerate}

Each regime differs in its identity conditions, persistence behavior, and
admissible relations.
Treating these differences as merely contextual or role-based distinctions
would make admissibility depend on interpretation, violating the requirement
of stable reference.

\subsection{Collapse Strategy and Admissibility}

To establish a lower bound, we consider whether any two of the regimes above
can be collapsed into a single identity regime without violating the assumptions of
Section~\ref{sec:assumptions}.
A collapse is admissible only if the resulting representation admits a single,
extension-stable identity regime and does not require hidden discriminators to
recover the original distinction.

For each possible collapse, at least one of the following outcomes arises:

\begin{itemize}
  \item \textbf{Identity instability:} the collapsed regime requires
        incompatible persistence criteria, so that identity varies across
        extensions.
  \item \textbf{Category error:} the collapsed regime admits relations that are
        structurally inappropriate, requiring interpretive rules to block
        them.
  \item \textbf{Hidden regime reintroduction:} the distinction is reintroduced
        via roles, flags, or contextual predicates, violating the
        no-smuggling criterion.
\end{itemize}

Because such mechanisms rely on interpretive agreement supplied by extension
layers, they are excluded by the stated optimization objective.

\subsection{Lower-Bound Result}

\begin{lemma}[Necessity of distinct regimes]
\label{lem:necessity}
Under the assumptions of Section~\ref{sec:assumptions}, no ontology whose stable partition
contains fewer than six distinct identity-and-persistence regimes can satisfy
the representational requirements of accountability while remaining neutral
and stable under incompatible extension.
\end{lemma}

\begin{proof}[Proof sketch]
Suppose an ontology realizes fewer than six identity-and-persistence regimes.
Then at least two of the regimes required for accountability are collapsed
into a single identity regime.
By the collapse analysis above, any such collapse either forces incompatible
identity criteria, admits structurally invalid relations, or requires internal
discriminators whose interpretation is supplied by extension layers.
Each case violates either stable reference or the neutrality constraint.
Therefore no ontology with fewer than six regimes can satisfy all stated
requirements simultaneously.
\end{proof}

\subsection{Interpretation of the Necessity Result}

The necessity result is conditional and structural.
It does not claim that fewer regimes are impossible in general, but that they
are incompatible with the optimization objective fixed in Section~\ref{sec:assumptions}.
Ontologies optimized for local agreement, negotiated semantics, or expressive
flexibility may legitimately employ fewer explicit distinctions by relying on
hidden or interpretive mechanisms.

Within the stated constraints, however, the result establishes a sharp lower
bound: any neutral substrate capable of supporting accountability under
persistent disagreement must realize at least six distinct
identity-and-persistence regimes.
The following section shows that this bound is tight by exhibiting a
construction that satisfies all requirements with a minimal set of six regimes.
  
% !TeX root = se200_identity_regimes.tex
\section{Sufficiency: A Six-Regime Construction}
\label{sec:sufficiency}

This section shows that the lower bound established in
Section~\ref{sec:necessity} is tight.
We exhibit a construction with a minimal set of six distinct
identity-and-persistence regimes that satisfies all structural requirements
stated in Section~\ref{sec:requirements}, while respecting the neutrality and
stability constraints fixed in Section~\ref{sec:assumptions}.
No additional regimes or hidden interpretive mechanisms are required.

\subsection{The Six Identity-and-Persistence Regimes}

We realize six pairwise-disjoint identity-and-persistence regimes.
For concreteness, each regime is defined solely by its 
identity conditions, persistence behavior, and admissible relations:

\begin{itemize}
  \item \textbf{K1 (Actors):} enduring entities capable of bearing obligations,
        rights, or responsibilities.
  \item \textbf{K2 (Loci/Assets):} enduring physical or operational entities that
        may be acted upon but do not bear obligations.
  \item \textbf{K3 (Instruments):} enduring authority-bearing structures that
        ground obligations or permissions independently of any particular
        occurrence.
  \item \textbf{K4 (Occurrences):} time-indexed actions or events whose identity
        is inseparable from temporal realization and provenance.
  \item \textbf{K5 (Scopes):} enduring applicability contexts that delimit where,
        to whom, or under what conditions instruments apply.
  \item \textbf{K6 (Indicators):} descriptive records or measurements that assert
        observed or derived facts without asserting causal or normative claims.
\end{itemize}

Each regime fixes a distinct identity criterion that is invariant under
incompatible extension.
No regime changes over time, and no regime’s identity depends on
interpretation supplied by analytic or application layers.

\subsection{Admissible Relations}

A minimal set of typed, directional relations suffices to connect the six
regimes and to satisfy all stated requirements.
Representative relations include:

\begin{itemize}
  \item authority grounding and delegation:
        $\textit{enacts}, \textit{issues} : \text{K1} \rightarrow \text{K3}$
  \item participation:
        $\textit{party-to} : \text{K1} \rightarrow \text{K3}$
  \item execution under authority:
        $\textit{occurs-under} : \text{K4} \rightarrow \text{K3}$
  \item involvement and action:
        $\textit{involves} : \text{K4} \rightarrow \text{K1}$,
        $\textit{acts-on} : \text{K4} \rightarrow \text{K2}$
  \item applicability:
        $\textit{applies-in} : \text{K3} \rightarrow \text{K5}$
  \item description:
        $\textit{measures} : \text{K6} \rightarrow
        (\text{K1} \mid \text{K2} \mid \text{K5})$
\end{itemize}

This relation set is not exhaustive, but illustrates that no additional
identity regimes are required to satisfy the structural constraints.
Temporal and provenance information attach only to K4 and K6 via attribute
schemas; no separate entities for time, provenance, or explanation are
introduced.

\subsection{Satisfaction of Structural Requirements}

We briefly verify that the construction satisfies the requirements of
Section~\ref{sec:requirements}:

\begin{itemize}
  \item \textbf{R1 (Stable identity):} each regime fixes a distinct, invariant
        identity criterion under extension.
  \item \textbf{R2 (Obligation-bearing):} obligations attach only to K1 via K3,
        preserving role stability.
  \item \textbf{R3 (Normative reference):} K3 grounds authority independently of
        execution, without causal or evaluative claims.
  \item \textbf{R4 (Time-indexed occurrence):} K4 captures temporal action without
        persistence ambiguity.
  \item \textbf{R5 (Applicability):} K5 provides first-class scope referents that
        support nesting and overlap without reclassification.
  \item \textbf{R6 (Descriptive records):} K6 represents indicators as stable
        referents distinct from occurrences, enabling longitudinal comparison
        without causal commitment.
\end{itemize}

All requirements are satisfied without introducing hidden regimes,
role-based typing, or interpretive predicates at the substrate level.

\subsection{Sufficiency Proposition}

\begin{proposition}[Sufficiency of six regimes]
\label{prop:sufficiency-six}
The six-regime construction defined above satisfies the neutrality, stability,
and accountability requirements fixed in
Sections~\ref{sec:assumptions} and~\ref{sec:requirements}.
\end{proposition}

\begin{proof}
Each requirement is satisfied by a distinct identity-and-persistence regime
with explicitly constrained admissible relations.
No requirement demands an additional regime, and no regime is redundant under
the stated optimization criterion.
Because the construction preserves stable reference under incompatible
extension and introduces no hidden distinctions, it is sufficient.
\end{proof}

\subsection{Tightness of the Bound}

Together with the necessity result of Section~\ref{sec:necessity}, the
sufficiency proposition establishes that six identity-and-persistence regimes
are both necessary and sufficient for a neutral accountability substrate under
the stated assumptions.
The bound is therefore tight with respect to the optimization objective of
interoperability under persistent disagreement.
   
% !TeX root = se200_identity_regimes.tex
\section{Interpretation of the Result}
\label{sec:interpretation}

This section clarifies how the lower- and upper-bound results should be
understood, and how they should not be understood.
The aim is not to defend the result against alternative modeling goals,
but to situate it relative to the explicit optimization criterion fixed
earlier in the paper.

\subsection{What the Result Establishes}

Taken together, the necessity and sufficiency results establish a
conditional optimality claim.
Under the assumptions of Section~\ref{sec:assumptions}, 
any ontology intended to function as a neutral substrate 
for accountability under persistent disagreement
must realize at least six distinct identity-and-persistence regimes.
Conversely, a construction with a minimal set of six such regimes suffices to meet
all stated representational requirements without introducing hidden
structure or interpretive commitments.

The result is therefore tight.
It does not merely show that six regimes are adequate, 
nor that fewer regimes are inadequate under particular conditions, 
but that six regimes are both necessary and sufficient 
under the stated optimization objective.

\subsection{Minimality Relative to Stability}

The result should not be read as a claim about simplicity in an intuitive
or engineering sense.
Ontologies with fewer explicit entity kinds may be easier to describe or extend
within a coordinated interpretive community.
However, when such reductions are achieved by encoding stability-critical
distinctions as roles, flags, or contextual predicates, total structural
complexity is not reduced.
Instead, it is displaced into extension layers whose interpretation cannot
be held fixed under disagreement.

In this paper, minimality is defined relative to stability.
A representation is minimal only if removing explicit structure does not
require reintroducing the same distinctions implicitly.
From this perspective, an ontology with fewer explicit kinds but greater
interpretive burden is not simpler, but less explicit.

\subsection{Why Hidden Regimes Are Excluded}

Hidden regimes are often effective where interpretation is shared,
enforced, or inexpensive to renegotiate.
They support flexibility and expressive economy within a single
governance framework.

The present analysis excludes such regimes not because they are incorrect,
but because they are incompatible with the stated objective.
When disagreement is persistent and uncoordinated, distinctions that are
not fixed at the substrate level cannot be relied upon to preserve identity
or admissible relations.
Stability then depends on interpretive convergence, which the substrate is
explicitly designed not to assume.

\subsection{Relation to Other Ontological Strategies}

Many upper ontologies and modeling frameworks adopt different optimization
objectives, including metaphysical coverage, expressive richness, or
alignment with natural language or scientific theory.
Such frameworks may introduce additional kinds or tolerate contextual
typing precisely in order to support richer modeling.

The result presented here neither subsumes nor contradicts those approaches.
It identifies a specific point in the design space: the minimal structure
required for a neutral, extension-stable substrate under persistent
disagreement.
Frameworks optimized for other goals may legitimately occupy different
points in that space.

\subsection{Conditional Scope of the Claim}

The lower-bound result is explicitly conditional.
If any assumption of Section~\ref{sec:assumptions} is rejected, 
such as neutrality, stability under incompatible extension, 
or the exclusion of hidden regimes, 
the optimization criterion is invalidated 
and the bound no longer applies.

Within the stated scope, however, the result is unavoidable.
Any attempt to reduce the number of regimes must either sacrifice stability,
reintroduce hidden distinctions, or weaken the neutrality constraint.
The six-regime structure thus emerges not as a design preference, but as a
structural consequence of the problem as posed.

\subsection{Implications}

The primary implication of this result is diagnostic rather than
prescriptive.
It provides a principled way to evaluate claims that an ontology supports
neutral interoperability under disagreement.
If such a framework relies on implicit roles or context-dependent typing
for stability-critical distinctions, the present result explains why those
claims cannot be sustained simultaneously under the stated constraints.

Conversely, for designers who share the stated optimization objective, the
result identifies which distinctions must be made explicit, and which may
safely be deferred to extension layers, in order to preserve neutrality and
stable reference across incompatible interpretations.

% !TeX root = se200_identity_regimes.tex
\section{Related Work}
\label{sec:related}

This paper is situated at the intersection of formal ontology, knowledge
representation, and the study of classification under disagreement.
Rather than surveying ontology engineering broadly, it situates the
present result with respect to prior work on neutrality, identity, and
structural stability.

\subsection{Classification, Disagreement, and Neutrality}

A substantial body of work has shown that classification systems embed
institutional, social, and political commitments rather than functioning
as neutral descriptive devices~\cite{bowker1999sorting,haslanger2012resisting}.
In scientific and policy contexts, disagreement is often persistent rather
than transient, reflecting incompatible evaluative frameworks rather than
resolvable empirical uncertainty~\cite{longino1990science}.

These observations motivate the need for representations that remain usable
without assuming interpretive convergence.
The present work adopts this motivation but addresses a distinct question:
what structural commitments are forced if disagreement is treated as a
permanent condition and interpretive commitments are externalized rather
than negotiated.

\subsection{Identity, Roles, and Ontological Methodology}

Formal ontology has long emphasized the role of identity conditions and
persistence criteria in category formation.
The OntoClean methodology~\cite{guarino2002evaluating} clarifies why roles are
anti-rigid and unsuitable as identity-defining categories~\cite{guarino1998roles},
and subsequent work on social and institutional entities has further
developed these distinctions~\cite{masolo2004wonderweb}.

This paper builds on these insights but applies them at the substrate level.
Rather than classifying domain entities, it treats identity-and-persistence
regimes as constraints imposed by the 
requirement of stable reference under incompatible extension.
The present analysis concerns a different level of ontological structure 
than role-based modeling, 
focusing on substrate-level identity-and-persistence regimes 
required for neutrality under disagreement.

\subsection{Upper Ontologies}

Upper ontologies such as BFO~\cite{smith2015basic}, DOLCE~\cite{gangemi2002dolce},
and UFO~\cite{guizzardi2005ontological} provide rich accounts of dependence,
endurants and perdurants, and social objects.
They differ substantially in commitments and optimization objectives, and
are not designed primarily as neutral substrates for interoperability under
persistent disagreement.

The present work does not compare expressivity or advocate alignment with any
particular upper ontology.
Instead, it derives lower-bound constraints that apply independently of such
frameworks, conditional on the neutrality and stability requirements adopted
here.

\subsection{Relation to Common Logic and ontology exchange}

The neutral substrate and derived kinds can be 
expressed in ISO/IEC 24707 Common Logic as a conservative theory, 
supporting interpreter-independent exchange 
without reliance on description-logic restrictions or fixed signatures.
This enables compatibility with ontology exchange workflows 
widely used in applied ontology and digital twin contexts.

\subsection{Causality, Explanation, and Infrastructure}

Separating descriptive structure from causal or explanatory claims is a
well-established principle in causal analysis~\cite{pearl2009causality}.
Similarly, research on large-scale information infrastructures emphasizes
the importance of long-term stability and interpretive restraint across
institutional boundaries~\cite{edwards2011infrastructure}.

These perspectives align with the pre-causal and pre-normative constraints assumed in this paper.
The ontology analyzed here is not an explanatory theory, but an
infrastructural substrate intended to support explanation, evaluation, and
policy analysis without embedding them.

\subsubsection{Relation to FAIR Data Principles}

The FAIR principles promote data reuse across communities and over time,
but do not specify the ontological constraints required for interoperability
under persistent disagreement.
In practice, reuse is often pursued through 
negotiated vocabularies or shared interpretations, 
which presuppose forms of convergence that may not be
available in legal, political, or analytic settings.

The results of this paper clarify a structural precondition for FAIR-style
reuse in pluralistic or institutionally heterogeneous contexts:
stable reference must be supported independently of 
causal, normative, or evaluative agreement.
The identity-and-persistence regimes derived here therefore constrain the
ontological substrates upon which FAIR-compliant systems implicitly rely
when reuse is expected across incompatible extensions.

% !TeX root = se200_identity_regimes.tex
\section{Conclusion}
\label{sec:conclusion}

This paper has addressed a structural question that arises when ontologies
are required to function as neutral substrates under persistent
disagreement.
Under explicit constraints of neutrality, stability under incompatible extension, 
and the exclusion of hidden or interpretive distinctions, we have
shown that a lower bound exists on the ontological structure 
required to support accountability.
Ontologies with fewer than six distinct identity-and-persistence regimes
cannot satisfy the stated neutrality and stability requirements;
a construction with exactly six regimes suffices without excess structure.

The result is conditional rather than prescriptive.
It does not claim that the identified structure 
is optimal for all applications, 
nor does it assess the adequacy of 
alternative modeling strategies for other objectives.
It establishes only that, under a particular and explicitly stated
optimization objective, certain distinctions are structurally unavoidable.

This work clarifies the design space for neutral substrates under persistent
disagreement.
Under the stated constraints, it shows that minimality is fixed by structural
requirements rather than by modeling preference.
Future work may examine how different optimization objectives yield
alternative bounds, or assess 
how the six-regime structure performs in
practical accountability applications.

\section*{Acknowledgements}

Portions of this work were developed
using computer-assisted tools during manuscript preparation.
Generative language models were used to assist with editing, formatting,
and consistency checking during manuscript preparation.
All conceptual framing, formal development, results, interpretations, and conclusions
are the author's own.
All generated suggestions were critically reviewed and validated, and
the author takes full responsibility for the content of this work.
     


\section*{Statements and Declarations}

\subsection*{Author Contributions}
The author was solely responsible for the conception, analysis, and writing of this manuscript.

\subsection*{Declaration of Conflicting Interest}
The author declares no potential conflicts of interest
with respect to the research, authorship,
and/or publication of this article.

\subsection*{Funding}
This research received no external funding.

\subsection*{Data Availability}
No datasets were generated or analyzed
during the current study.

\subsection*{Ethical Approval}
Not applicable.

\subsection*{Consent to Participate}
Not applicable.

\subsection*{Consent for Publication}
Not applicable.

\appendix
\clearpage

% !TeX root = se200_identity_regimes.tex
\clearpage
\section*{Appendix A. Glossary of Terms}
\addcontentsline{toc}{section}{Appendix A. Glossary of Terms}

This appendix defines key terms used throughout the paper.

% !TeX root = se200_identity_regimes.tex
% =============================================================================
% Glossary: Identity Regimes and Structural Constraints
% =============================================================================

\section*{Glossary}
\addcontentsline{toc}{section}{Glossary}

\begin{description}[style=nextline,leftmargin=1.6cm]

      \item[Accountability Substrate]
            A representational layer intended to support attribution, inspection,
            and comparison of responsibility-related structures
            without embedding causal, normative, or evaluative commitments.
            In this paper, accountability is treated structurally rather than interpretively.

      \item[Applicability Context]
            An enduring referent that delimits where, to whom, or under what conditions
            an authority-bearing structure applies.
            Applicability contexts persist independently of specific occurrences
            and must support nesting and overlap.

      \item[Hidden Regime]
            An identity- or persistence-relevant distinction that is not represented
            as an explicit ontological kind,
            but is instead encoded via roles, flags, contextual predicates,
            or interpretive conventions supplied by extension layers.
            Hidden regimes are excluded under the stability optimization adopted here.

      \item[Identity-and-Persistence Regime]
            A specification of the conditions under which an entity remains the same
            entity over time and across incompatible extensions,
            together with the relations it may admit without reclassification.
            Each regime defines a distinct mode of reference stability.

      \item[Interpretive Extension]
            Any explanatory, normative, analytic, or evaluative structure
            added atop the substrate that supplies meaning, causation,
            or judgment without altering substrate identity conditions.

      \item[Lower-Bound Result]
            A necessity claim establishing that no representation
            with fewer than a specified number of identity-and-persistence regimes
            can satisfy the stated requirements under the adopted optimization criterion.

      \item[Neutrality Constraint]
            The requirement that the substrate itself
            remain pre-causal and pre-normative,
            permitting incompatible interpretations
            without requiring revision of entity identity or structure.

      \item[Optimization Criterion]
            The objective function fixed for this paper:
            to minimize explicit ontological structure
            subject to neutrality, stable reference,
            and the exclusion of hidden identity regimes.
            Minimality is defined relative to stability, not syntactic brevity.

      \item[Persistent Disagreement]
            A condition in which interpretive, normative,
            or analytic incompatibilities are expected to endure over time,
            rather than converge through negotiation, evidence, or governance.

      \item[Stable Reference]
            The property that entities remain referable
            as the same entities across time and across incompatible extensions,
            without reclassification or reinterpretation of identity criteria.

      \item[Sufficiency Construction]
            An explicit ontological partition that satisfies all stated requirements
            using exactly the minimal number of identity-and-persistence regimes
            established by the lower-bound result.

      \item[Tight Bound]
            A result showing that the same number of regimes
            is both necessary and sufficient
            under the stated assumptions and optimization objective.

      \item[Time-Indexed Occurrence]
            An entity whose identity is inseparable from its temporal realization
            and provenance.
            Occurrences do not persist beyond their happening
            and are distinct from enduring entities.

\end{description}
   % Glossary

\ifShowAnnotations
  \clearpage
  % !TeX root = se200_identity_regimes.tex

\clearpage
\section*{Appendix B. Requirement--Discharge Trace (P0)}
\addcontentsline{toc}{section}{Appendix B. Requirement--Discharge Trace (P0)}

% Mechanical traceability matrix. No new claims.
\begingroup
\small
\setlength{\tabcolsep}{4pt}
\renewcommand{\arraystretch}{1.15}


\subsection*{B.1 Global Proof Contract Requirements}
\begin{tabular}{p{0.13\linewidth} p{0.50\linewidth} p{0.15\linewidth} p{0.15\linewidth} p{0.05\linewidth}}
  \textbf{REQ ID} & \textbf{Requirement}                                                                                                  & \textbf{Introduced} & \textbf{Discharged}                             & \textbf{OK} \\
  \hline
  PROOF           & All claims MUST follow from explicit definitions, stated assumptions, and logical inference.                          & Contract            & \ref{sec:requirements}, \ref{sec:impossibility} & Yes         \\
  WELL-FORMED     & All formal objects MUST be admissible and well-formed with respect to the assumed ontology and substrate constraints. & Contract            & \ref{sec:requirements}, \ref{sec:impossibility} & Yes         \\
  PC              & The substrate MUST be pre-causal.                                                                                     & Contract            & \ref{sec:requirements}, \ref{sec:impossibility} & Yes         \\
  NEUTRAL         & The substrate MUST not encode normative, evaluative, or policy judgments.                                             & Contract            & \ref{sec:requirements}, \ref{sec:impossibility} & Yes         \\
  ID-INVAR        & Entity existence and identity conditions MUST be invariant across admissible frameworks within the domain.            & Contract            & \ref{sec:requirements}, \ref{sec:impossibility} & Yes         \\
  ID              & Entity identity MUST be established prior to explanation or interpretation.                                           & Contract            & \ref{sec:requirements}                          & Yes         \\
  MINIMAL         & Substrate MUST include only entities and relations necessary for accountability analysis, and no more.                & Contract            & \ref{sec:implications}                          & Yes         \\
  SCOPE-GUARD     & Paper MUST NOT argue all ontologies should be neutral; only that neutral substrates have categorical constraints.     & Contract            & \ref{sec:requirements}                          & Yes         \\
  DEP             & Direction of dependence MUST be: Identity precedes explanation; explanation precedes interpretation.                  & Contract            & \ref{sec:requirements}, \ref{sec:impossibility} & Yes         \\
\end{tabular}


\subsection*{B.2 Neutrality Requirements (Section~\ref{sec:requirements})}
\begin{tabular}{p{0.12\linewidth} p{0.40\linewidth} p{0.15\linewidth} p{0.15\linewidth} p{0.05\linewidth}}
  \textbf{REQ ID} & \textbf{Requirement}                                                                   & \textbf{Derives From} & \textbf{Discharged By}       & \textbf{OK} \\
  \hline
  INC             & Substrate MUST exclude causal and normative observables from its Level of Abstraction. & PC, NEUTRAL           & \ref{subsec:interpretive-nc} & Yes         \\
  EXT             & Substrate MUST remain stable under extension by incompatible interpretive frameworks.  & DEP                   & \ref{subsec:extension}       & Yes         \\
\end{tabular}

\noindent\textbf{Handoff:} \OBS{REQ-TO-IMP}{Section~\ref{sec:requirements} defines neutrality
  such that violating pre-causality or pre-normativity entails failure of extension stability,
  enabling the impossibility argument in Section~\ref{sec:impossibility}.}


\subsection*{B.3 Impossibility Discharge (Section~\ref{sec:impossibility})}
\begin{tabular}{p{0.55\linewidth} p{0.33\linewidth} p{0.07\linewidth}}
  \textbf{Claim / Obligation}                                         & \textbf{Discharged By}    & \textbf{OK} \\
  \hline
  INC/EXT incompatible with normative commitments                     & \ref{subsec:normative}    & Yes         \\
  INC/EXT incompatible with causal commitments                        & \ref{subsec:causal}       & Yes         \\
  Reification does not provide a middle ground (boundary only)        & \ref{subsec:reification}  & Yes         \\
  Formal contradiction from requirements to impossibility proposition & \ref{subsec:formal-proof} & Yes         \\
\end{tabular}


\subsection*{B.4 Derived Design Constraints (Section~\ref{sec:implications})}
\begin{tabular}{p{0.38\linewidth} p{0.50\linewidth} p{0.07\linewidth}}
  \textbf{Derived Constraint}                                                           & \textbf{Derived From}                                                          & \textbf{OK} \\
  \hline
  Exclude causal, normative, evaluative conclusions from substrate assertions           & INC + EXT (\ref{sec:requirements}) and contradiction (\ref{sec:impossibility}) & Yes         \\
  Provide entity reference, identity criteria, disjointness, and persistence conditions & ID (Contract) + LoA restriction (\ref{sec:requirements})                       & Yes         \\
  Externalize interpretation to higher LoO layers                                       & DEP (Contract) + stability requirement (\ref{sec:requirements})                & Yes         \\
  Maintain monotonicity under incompatible extensions                                   & EXT (\ref{sec:requirements}) + impossibility (\ref{sec:impossibility})         & Yes         \\
  Use reification only as boundary mechanism (no endorsement)                           & \ref{subsec:reification} + INC/EXT framing                                     & Yes         \\
\end{tabular}

\subsection*{B.5 Integrity Check (Mechanical)}
\begin{tabular}{p{0.65\linewidth} p{0.30\linewidth}}
  \textbf{Check}                                                    & \textbf{Result}  \\
  \hline
  REQ introduced outside Contract or Section~\ref{sec:requirements} & None             \\
  REQ without discharge path                                        & None             \\
  Annotated claims in Intro/Related/Conclusion                      & None (by design) \\
\end{tabular}

\endgroup
   % Trace (internal)
\fi

\bibliographystyle{plainnat}
\bibliography{bibliography/bibliography}
\end{document}