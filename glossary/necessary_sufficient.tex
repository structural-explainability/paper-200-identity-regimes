% !TeX root = se200_identity_regimes.tex
% =============================================================================
% Glossary: Identity Regimes and Structural Constraints
% =============================================================================

\section*{Glossary}
\addcontentsline{toc}{section}{Glossary}

\begin{description}[style=nextline,leftmargin=1.6cm]

      \item[Accountability Substrate]
            A representational layer intended to support attribution, inspection,
            and comparison of responsibility-related structures
            without embedding causal, normative, or evaluative commitments.
            In this paper, accountability is treated structurally rather than interpretively.

      \item[Applicability Context]
            An enduring referent that delimits where, to whom, or under what conditions
            an authority-bearing structure applies.
            Applicability contexts persist independently of specific occurrences
            and must support nesting and overlap.

      \item[Hidden Regime]
            An identity- or persistence-relevant distinction that is not represented
            as an explicit ontological kind,
            but is instead encoded via roles, flags, contextual predicates,
            or interpretive conventions supplied by extension layers.
            Hidden regimes are excluded under the stability optimization adopted here.

      \item[Identity-and-Persistence Regime]
            A specification of the conditions under which an entity remains the same
            entity over time and across incompatible extensions,
            together with the relations it may admit without reclassification.
            Each regime defines a distinct mode of reference stability.

      \item[Interpretive Extension]
            Any explanatory, normative, analytic, or evaluative structure
            added atop the substrate that supplies meaning, causation,
            or judgment without altering substrate identity conditions.

      \item[Lower-Bound Result]
            A necessity claim establishing that no representation
            with fewer than a specified number of identity-and-persistence regimes
            can satisfy the stated requirements under the adopted optimization criterion.

      \item[Neutrality Constraint]
            The requirement that the substrate itself
            remain pre-causal and pre-normative,
            permitting incompatible interpretations
            without requiring revision of entity identity or structure.

      \item[Optimization Criterion]
            The objective function fixed for this paper:
            to minimize explicit ontological structure
            subject to neutrality, stable reference,
            and the exclusion of hidden identity regimes.
            Minimality is defined relative to stability, not syntactic brevity.

      \item[Persistent Disagreement]
            A condition in which interpretive, normative,
            or analytic incompatibilities are expected to endure over time,
            rather than converge through negotiation, evidence, or governance.

      \item[Stable Reference]
            The property that entities remain referable
            as the same entities across time and across incompatible extensions,
            without reclassification or reinterpretation of identity criteria.

      \item[Sufficiency Construction]
            An explicit ontological partition that satisfies all stated requirements
            using exactly the minimal number of identity-and-persistence regimes
            established by the lower-bound result.

      \item[Tight Bound]
            A result showing that the same number of regimes
            is both necessary and sufficient
            under the stated assumptions and optimization objective.

      \item[Time-Indexed Occurrence]
            An entity whose identity is inseparable from its temporal realization
            and provenance.
            Occurrences do not persist beyond their happening
            and are distinct from enduring entities.

\end{description}
